\documentclass[a4paper,openright,12pt]{book}
\usepackage[utf8]{inputenc}
\usepackage[T1]{fontenc}
\usepackage[spanish,es-tabla]{babel}
\usepackage{graphicx}
\usepackage{appendix}
%\usepackage{caption}
\usepackage{epstopdf}
\usepackage[usenames,dvipsnames,svgnames,table]{xcolor}
%\usepackage{subcaption}
\usepackage{fancyhdr}
\usepackage{fullpage}
\usepackage{anysize}		
\usepackage{vmargin}
\usepackage{amsmath}
\usepackage{amssymb}
\usepackage{subfigure}
%\usepackage{subfig}
\usepackage{float}
\usepackage{subfloat}
%\usepackage{caption}
\usepackage{enumerate}


\setmargins{2.5cm}       % margen izquierdo
{1.5cm}                        % margen superior
{16.59cm}                      % anchura del texto
{22.94cm}                    % altura del texto
{10pt}                           % altura de los encabezados
{1cm}                           % espacio entre el texto y los encabezados
{0pt}                             % altura del pie de página
{1cm}                           % espacio entre el texto y el pie de página

%PIE DE PAGINA
%\pagestyle{empty} quita enumeracion de la hoja.
\pagestyle{fancy}

\lhead{}

\chead{} % texto centro de la cabecera
\lhead{} % texto izquierdo de la cabecera
\rhead{} % texto derecho de la cabecera
\lfoot{} % texto izquierda del pie
\cfoot{} % imagen centro del pie
\rfoot{\thepage} % texto derecha del pie
\renewcommand{\headrulewidth}{0.4pt} % grosor de la línea de la cabecera
\renewcommand{\footrulewidth}{0.4pt} % grosor de la línea del pie
%\end{comment}
%------------------------------------------------------
\begin{document}
\thispagestyle{empty}

\begin{titlepage}

\begin{center}
\vspace*{-1in}
\begin{figure}[htb]
\begin{center}
\includegraphics[width=8cm]{./img/usm.jpg}
\end{center}
\end{figure}

DEPARTAMENTO DE INDUSTRIAS \\
\vspace*{0.6in}
\begin{large}
TESIS DE MAGISTER:\\
\end{large}
\vspace*{0.2in}
\begin{Large}
\textbf{PROPUESTA DE UN MODELO PARA EL DESARROLLO AGIL DE APLICACIONES EN DISPOSITIVOS MÓVILES: EL FUTURO DE LA INDUSTRIA BANCARIA} \\
\end{Large}
\vspace*{0.3in}
\begin{large}
Javier Silva Pacheco\\
\end{large}
\vspace*{0.3in}
\rule{80mm}{0.1mm}\\
\vspace*{0.1in}
\begin{large}
Profesor: \\
XXXXX X. XXXXXXXX \\
\end{large}
\end{center}

\end{titlepage}

\newpage
$\ $
\thispagestyle{empty} % para que no se numere esta pagina

\chapter*{}
\pagenumbering{Roman} % para comenzar la numeracion de paginas en numeros romanos
\begin{flushright}
\textit{Dedicado a \\
mi familia}
\end{flushright}

\chapter*{Agradecimientos} % si no queremos que añada la palabra "Capitulo"
\addcontentsline{toc}{chapter}{Agradecimientos} % si queremos que aparezca en el índice
\markboth{AGRADECIMIENTOS}{AGRADECIMIENTOS} % encabezado 

¡Muchas gracias a todos!


\chapter*{Resumen} % si no queremos que añada la palabra "Capitulo"
\addcontentsline{toc}{section}{Resumen} % si queremos que aparezca en el índice
\markboth{RESUMEN}{RESUMEN} % encabezado

Este trabajo presenta una propuesta metodológica que permite hacer mas eficiente el proceso de desarrollo de aplicaciones móviles en la industria bancaria. El objetivo de esta investigación es proponer un modelo de desarrollo de software enfocado en los distintos dominios de productos y servicios financieros, generando mejoras incrementales de valor para los clientes, en periodos fijos de tiempo. Para lograrlo, se toman como base las metodologías ágiles de desarrollo de software, donde se propone una visión desde el producto para la configuración de los equipos y el manejo de expectativas de cara a los distintos stakeholders del proyecto, afectando de manera positiva las métricas de leadtime. La propuesta permite escalar el trabajo de los equipos de desarrollo de manera efectiva, tomando en cuenta las necesidades del negocio y la mejor experiencia de usuario para el cliente final. Explicitando, que es posible atender de manera gradual y paralela a las distintas áreas comerciales del banco, las cuales tienen la necesidad de estar presentes en los canales digitales debido a la penetración de dispositivos móviles y el uso que se proyecta para el futuro. Todo lo anterior es posible lograrlo, tomando en cuenta una estructura liviana y fomentando el empoderamiento de los equipos y el trabajo colaborativo.

\chapter*{Highlights} % si no queremos que añada la palabra "Capitulo"
\addcontentsline{toc}{section}{Highlights} % si queremos que aparezca en el índice
\markboth{HIGHLIGHTS}{HIGHLIGHTS} % encabezado

\begin{itemize}
    \item Este método es una propuesta que ayuda a la banca en sus procesos de transformación digital.
    \item 
    \item Castilla y León.
    
\end{itemize}


\tableofcontents % indice de contenidos

\cleardoublepage
\addcontentsline{toc}{chapter}{Lista de figuras} % para que aparezca en el indice de contenidos
\listoffigures % indice de figuras

\cleardoublepage
\addcontentsline{toc}{chapter}{Lista de tablas} % para que aparezca en el indice de contenidos
\listoftables % indice de tablas

\chapter{Introducción}\label{cap.introduccion}
\pagenumbering{arabic}
Érase una vez...
\section{sección1}
Bla bla bla
\subsection{subsección1}
Ble ble ble
\subsubsection{subsubsección1}
Bli bli bli
\paragraph{párrafo1}
Blo blo blo

\chapter{Nudo}\label{cap.nudo}
La historia continúa con...

\chapter{Desenlace}\label{cap.desenlace}
El final de la historia es sorprendete...

\appendix
\chapter{Más cosas}\label{aped.A}
Aún faltan cosas por decir.

\chapter{Y más cosas aún}\label{aped.B}
Y más cosas aún.

\cleardoublepage
\addcontentsline{toc}{chapter}{Bibliografía}
\bibliographystyle{acm} % estilo de la bibliografía.
\bibliography{yyyy} % yyyy.bib es el fichero donde está salvada la bibliografía.

\end{document}
